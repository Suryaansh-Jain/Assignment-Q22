\documentclass[journal,12pt,twocolumn]{IEEEtran}

\usepackage{enumitem}
\usepackage{amsmath}
\usepackage{amssymb}
\usepackage{graphicx}


\title{Assignment 2 \\}
\author{Suryaansh Jain \\ \normalsize cs21btech11057 \\}


\begin{document}
	% The title
	\maketitle
	
	% The question
	\textbf{Question 21(a)} 
	A product can be manufactured at a total cost $C(x) = \frac{x^{2}}{100}+100x+40$, where x
is the number of units produced. The price at which each unit can be sold is given
by $P$ = $200 - \frac{x}{400}$. Determine the production level x at which the profit is
maximum. What is the price per unit and total profit at the level of production? \\
	
	
	% The solution
	\textbf{Solution.}		
	\begin{align}
	    &\text{Let\ the\ total\ price\ }p(x) = P.x \\
		&\implies \frac{p(x)}{x} = 200 - \frac{x}{400} \\
		&c(x) = \frac{x^{2}}{100} + 100x + 40 \\
		&Profit = p(x) - c(x)
	\end{align}
	
	For maximum profit $\frac{dProfit}{dx} = 0$
	
	\begin{align}
	    &\implies 100 - \frac{x}{40} = 0 \\
	    &\implies x = 4000
	\end{align}
	
	The total production level $x = 4000$.
	
	\begin{align}
        &Price\ per\ unit =  P(x)  = 190 \\
        &Profit= 199960
	\end{align}
	
	Total profit = 199960
		
	\begin{center} \\
\begin{tabular}{|l|l|l|}
    \hline
    Symbol & Description & Value \\
    \hline
    $P$ & Price per unit & \rupee 190 \\
    $P(x)$ & Total price & \rupee  760000\\
    $C(x)$ & Total cost & \rupee 560040\\
    $Profit$ & Total profit & \rupee 199960 \\
    \hline
\end{tabular}
\end{center}
	
\begin{figure}[!ht]
    \centering
    \includegraphics[width=\columnwidth]{main.png}
    \caption{Graph shows Total Profit, Total cost, Total profit with x}
    \label{fig:graph}
\end{figure}

\end{document}
